\intro

\textbf{Актуальность курсовой работы.} Трудовые споры являются всегда актуальной темой в любой стране, в том числе и в России. Конфликты между работниками и работодателями неизбежны, и они могут возникнуть по самым разным причинам. Но ясно одно – когда стороны не могут решить конфликт мирным путем, они обращаются в государственные органы и суды за защитой своих прав.

\textbf{Теоретическую базу и степень научной разработанности} данного исследования составили труды Ю.П. Орловского, Филиповой И.А.,  Костяна И.А., Гусова К.Н., Толкуновой В.Н., Петрова А.Я., Буянова М.О. 

\textbf{Объект:} общественные отношения между работодателем и работником в сфере трудовых споров.

\textbf{Предмет:} общественные отношения с сфере исполнения решений судов по трудовым спорам и досудебных стадий. 

\textbf{Цель исследования: } установить роль трудовых споров в общественных отношениях.

\textbf{Задачи исследования: }

- определение понятия трудового спора;

- анализ основных видов трудовых споров;

- изучение процедуры рассмотрения трудовых споров в России;

- анализ роли судебной власти в урегулировании трудовых споров;

- анализ зарубежных систем регулирования трудовых споров;

- исторический анализ решений трудовых споров в России.

\textbf{Методологическую основу} исследования составили следующие методы исследования: анализ, синтез, дедукция, индукция, аналогия, исторический, сравнительный методы научного анализа.

\textbf{Структура работы: } Цель и задачи исследования определили следующую структуру работы: Введение; Глава I: \enquote{Трудовые споры: основные принципы, практика и пути совершенствования}: \S1. Категории и виды трудовых споров; \S2. Процедура рассмотрения трудовых споров; \S3. Примеры урегулирования трудовых споров в суде; \S4. Проблемы, возникающие в трудовых спорах; Глава II: \enquote{Исследование путей реформирования трудовых споров в различных странах мира, с учётом опыта их реформ}: \S1. Трудовые споры в США; \S2. Трудовые споры в Японии; \S3. Трудовые споры во Франции; Глава III:  \enquote{Исторический анализ трудовых споров в России}: \S1. История решения трудовых споров в России; \S2. Решение трудовых споров в Древней Руси; \S3. Решение трудовых споров в Российской Империи; \S3. Решение трудовых споров в СССР; Заключение; Список использованной литературы.

\chapter{ТРУДОВЫЕ СПОРЫ: ОСНОВНЫЕ ПРИНЦИПЫ, ПРАКТИКА И ПУТИ СОВЕРШЕНСТВОВАНИЯ}

Трудовой спор – это конфликт между работодателем и работником, который возникает в процессе трудовых отношений и связан с нарушением трудовых прав и интересов работника или работодателя.

В России трудовые споры регулируются Трудовым Кодексом статьей 387 
«Порядок рассмотрения индивидуальных трудовых споров и коллективных трудовых споров».

Несмотря на наличие нормативных актов, регулирующих трудовые отношения 
между работником и работодателем, существуют случаи, когда возникают споры, 
требующие разрешения. Трудовые споры могут возникнуть по разным причинам: 
от несоблюдения трудового законодательства до разногласий в оплате и условиях труда.
 Цель данной курсовой работы - изучить категории и виды трудовых споров.

Трудовые споры могут быть разделены на следующие категории:

1. Персональные споры. Они возникают между отдельными работниками и работодателями в связи с нарушением трудовых прав работника, оплатой труда, работой в неполном объеме и так далее.

2. Коллективные споры. Они связаны с нарушением прав и интересов работников по общим вопросам труда, правового статуса коллектива и т.д. К этой категории относятся забастовки, массовые протесты и другие формы коллективного выражения мнения.

3. Юридические споры. Они возникают между работником и работодателем по поводу интерпретации и применения законодательных актов, локальных нормативных актов, трудового договора и т.д.

4. Экономические споры. 
Они возникают в связи с экономическими проблемами предприятия, 
предполагающими влияние на трудовые отношения, в том числе сокращение 
численности работников, изменение условий оплаты труда и так далее.

5. Медиационные споры. Они возникают в процессе медиации между работником и работодателем в целях разрешения конфликтных ситуаций. Для справки «Медиация» – это способ досудебного урегулирования споров и безболезненное решение конфликтных ситуаций с привлечением независимого специалиста, который поможет вам в переговорах.

Трудовые споры можно разделить на несколько видов:

1. Индивидуальные трудовые споры -– конфликты между работником и работодателем, 
которые возникают в связи с нарушением трудовых прав и интересов конкретного работника.

2. Коллективные трудовые споры -- конфликты между работодателем и коллективом работников, которые связаны с нарушением прав и интересов группы работников.

3. Профессиональные споры -- конфликты между профсоюзами и работодателями, связанные с вопросами оплаты труда, условий труда, занятости и другими существенными условиями труда.

В зависимости от конкретной ситуации и причины возникновения спора, можно выделить следующие подвиды трудовых споров:

1. Несогласие с условиями труда. Этот вид спора возникает при несогласии работника с изменениями условий труда, такими как увеличение рабочего времени, снижение заработной платы и так далее.

2. Неуплата заработной платы. Этот вид спора возникает, когда работник не получает заработную плату или получает ее в меньшем размере, чем предусмотрено трудовым договором и законодательством.

3. Нарушение договоренностей. Этот вид спора возникает при нарушении договоренностей между работником и работодателем, касающихся оплаты труда, условий труда и так далее.

4. Увольнение. Этот вид спора возникает при несогласии работника с причинами увольнения, такими как необоснованный увольнение, нарушение процедуры увольнения и так далее.

5. Дискриминация. Этот вид спора возникает при дискриминации работника по полу, возрасту, национальности и другим основаниям.

6. Отказ в предоставлении отпуска. Этот вид спора возникает при отказе работодателя предоставить отпуск, которые может быть предоставлен работнику в соответствии с трудовым законодательством и трудовым договором.

7. Моббинг. Этот вид спора возникает при моббинге работников, т.е. умы-шленном давлении на работника со стороны коллег или начальства.

По итогу можно сказать, что трудовые споры возникают по разным причинам и могут иметь различную форму. Но независимо от их вида и категории, важно разрешать их в соответствии с законодательством и уважением к правам работников и работодателей. При возникновении трудового спора следует обращаться к компетентным органам и лицам для разрешения этого спора.

\section{Процедура рассмотрения трудовых споров }
Рассмотрение трудовых споров проводится в соответствии с порядком, установленным законодательством, а именно ТК РФ Глава 60. Рассмотрение и разрешение индивидуальных трудовых споров.

Первым шагом является обращение работника или работодателя в комиссию по рассмотрению трудовых споров. Комиссия состоит из представителей работников и работодателя, а также представителей профсоюзов при наличии последних на предприятии. Комиссия должна рассмотреть конфликт в течение 10 дней и вынести решение, которое может быть обжаловано в судебном порядке.

Если конфликт не урегулирован в комиссии, то работник или работодатель могут обратиться в суд за защитой своих прав. Суд рассматривает трудовой спор в обязательном порядке и выносит соответствующее решение. 

Один из наиболее важных вопросов, связанных с трудовыми отношениями между работодателем и работником, это рассмотрение трудовых споров, которые могут возникнуть в процессе трудовой деятельности. Росийское законодательтво устанавливает специальные процедуры для решения таких споров, которые можно разделить на три основных этапа: до судебного разбирательства, во время судебного разбирательства и после его окончания. Данный курсовая работа рассматривает процедуру рассмотрения трудовых споров в России.

Первый этап в процедуре рассмотрения трудовых споров - это попытка решения спора внесудебным путем. Решение спора внесудебным путем наиболее часто применяется в случаях легких нарушений трудового договора, таких как нарушение графика работы, не выдача на время заработной платы и т.д.

Для решения трудовых споров внесудебным путем может быть использован механизм коллективных переговоров между работодателем и трудовым коллективом или юридический консультацииспециалиста. В случае несогласия между работодателем и работником может быть организовано собрание трудового коллектива или создана комиссия по разрешению трудовых споров, состоящая из представителей работодатель и работников. Если же эти меры не приводят к консенсусу, работник может обратиться в организацию, которая занимается защитой прав работников – профсоюз.

Если внесудебное разрешение трудового спора не удалось, работник может обратиться в судебную инстанцию. В зависимости от сложности и значимости спора разбирательство может быть произведено как в рабочей комиссии, так и в суде. Трудовой спор может быть рассмотрен по заявлению работника или работодателя, либо по инициативе контролирующих органов, если были выявлены нарушения законодательства о труде.

Помимо федеральных судов, где судебной властью обладает судья, в России имеются Арбитражные суды, где в процессе рассмотрения участвует коллегия судей. Рабочие комиссии также составляются либо из специалистов в области трудового права, либо из представителей работодателя и работников, выдвигаемых соответственно штатом работников и руководством.

После окончания судебного разбирательства и принятия решения по трудовому спору стороны обязаны выполнить решение, вынесенное судом. Как правило, решение суда является окончательным и не подлежит обжалованию. Тем не менее, в некоторых случаях возможны апелляции.

Таким образом, процедура рассмотрения трудовых споров в России пре-дусматривает ряд механизмов, начиная от внесудебного разрешения до обращения в рабочую комиссию или в судебную инстанцию. Работники и работодатели могут воспользоваться различными способами решения споров, в зависимости от сложности и значимости трудовых оспариваний. Необходимо отметить, что правильное рассмотрение трудовых споров очень важно для поддержания стабильности трудовых отношений и защиты прав всех сторон.

Суды имеют важную роль в урегулировании трудовых споров в России. Они рассматривают все трудовые споры, которые не были урегулированы в комиссии по рассмотрению трудовых споров. Так же стоит отметить, что споры между работодателем и работником могут быть рассмотрены в районном, городском или областном суде, в зависимости от суммы, на которую подаётся иск.

Суды выносят решение в соответствии с законодательством в первую очередь он руководствуетя ТК РФ Глава 60. Рассмотрение и разрешение индивидуальных трудовых споров, а также другими нормативными актами, которые регулируют трудовые отношения между работником и работодателем основываясь на фактах, представленных сторонами, но, суд может удовлетворить иск работника, либо работодателя, а также отклонить иск. Решения суда имеют обязательную юридическую силу и должны быть обязательно выполнены сторонами. Конечно стоит сказать, что решения суда являются обязательными для выполнения сторонами трудового спора. Если одна из сторон не согласна с решением суда, то она может обжаловать его в вышестоящую инстанцию.

Следует отметить, что процесс рассмотрения трудовых споров может занимать много времени. В связи с этим, многие организации пытаются решать споры в рамках досудебного порядка, выходя на переговоры со сторонами и пытаясь найти компромиссное решение.

\section{Примеры урегулирования трудовых споров в суде}

Для более полного понимания роли судов в урегулировании трудовых споров, приведем несколько примеров.

Пример №1. Работник уволился из своей организации, которая долгое время не выплачивала ему заработную плату. Работник обратился в суд с иском о взыскании неуплаченной суммы. Суд удовлетворил иск, вынес решение о выплате работнику неустойки и неуплаченной заработной платы.

Пример №2. Работник обратился в суд с иском о восстановлении на работе после того, как его уволили без уважительной причины. Суд признал, что работник был уволен без уважительной причины и вынес решение о его восстановлении на работе.

Пример №3. Работник подал в суд иск о неустойке за несвоевременную выплату заработной платы. Суд принял решение об удовлетворении иска и вынес решение о выплате работнику неустойки за нарушение сроков выплаты заработной платы.

В итоге мы выяснили, что трудовые споры могут возникнуть в любой момент при выполнении трудовых обязанностей. Суды играют важную и необходимую роль в урегулировании таких споров, рассматривая дела в соответствии с законодательством и вынося решения, которые имеют обязательную юридическую силу. Однако, стоит помнить, что процесс рассмотрения трудовых споров может занимать довольно много времени и потребовать значительных затрат. Поэтому как показывает практика, лучше пытаться решать споры в рамках досудебного порядка, выходя на переговоры со сторонами и пытаясь найти компромиссное решение.

\section{Проблемы, возникающие в трудовых спорах}

Трудовые споры являются традиционной составляющей современной экономики и представляют собой противоречия между работодателями и работниками, связанные с условиями труда, материальным вознаграждением, социальными гарантиями и другими вопросами. Решение этих проблем является важным аспектом гарантии прав работников и стабильности взаимоотношений в трудовой сфере.

Одной из основных проблем трудовых споров является отсутствие совместимости интересов между работодателями и работниками. В современном мире большое количество компаний и организаций ориентированы на прибыль, их целью является максимизация производительности и прибыли. Большинство работодателей властны и агрессивны, они считают, что работники должны следовать правилам и порядку, которые определяет работодатель. В результате этого у работников возникают недовольства по ряду вопросов, связанных с трудовым правом, охраной труда, социальными гарантиями и другими аспектами, что ведет к возникновению трудовых споров.

Еще одной проблемой является недостаточность профессиональных конфликтологов и адвокатов, специализирующихся на решении трудовых споров. В большинстве случаев, когда возникает трудовой спор, работник и работодатель могут не иметь достаточного опыта или знаний для решения этой проблемы. В результате этого, решение спора может затянуться, а иногда и вообще не быть найдено во благо обеих сторон.

Проблемой в решении трудовых споров также является общее состояние экономики и занятости на определенной территории. В периоды экономического кризиса и рецессии работодатели могут принимать меры, связанные с уменьшением затрат на персонал, что ведет к сокращению штатов и падению зарплат работников. Возникает коллизия интересов работодателей и работников, которую можно решить только с помощью долгосрочных экономических мер, направленных на стимулирование роста экономики и улучшение занятости.

Кроме этого, важная проблема, способной обострить трудовые споры, является несправедливое увольнение работника или принятие другого решения, связанного с персоналом, которое не соответствует законам и, как следствие, правам работника. Такое решение могут принять работодатели в условиях экономических трудностей или с целью повышения эффективности производства.

Резюмируя, можно сказать, что трудовые споры могут быть решены только с помощью компромиссов и совместного соглашения между работодателями и работниками. Иногда к решению спора могут быть привлечены третьи стороны, например, арбитражные суды или профсоюзы. Главным же аспектом при решении трудовых споров является то, что стороны должны понимать и уважать друг друга, а не ставить свои интересы выше интересов другой стороны.

\chapter{ИССЛЕДОВАНИЕ ПУТЕЙ РЕФОРМИРОВАНИЯ ТРУДОВЫХ СПОРОВ В РАЗЛИЧЫХ СТРАНАХ МИРА, С УЧЁТОМ ОПЫТА ИХ РЕФОРМ}

\section{Трудовые споры в США}
В США очень распространены коллективные и групповые трудовые споры. Они рассматриваются в специальных судах, которые относятся к федеральной власти. Однако, процедуры таких судов часто слишком долгие и затратные.

В связи с этим, существует возможность арбитражного разрешения трудовых споров, когда спор между работодателем и работником разрешается без привлечения суда, а через независимую третью сторону - арбитражёра. При этом решение арбитража является окончательным и обязательным для исполнения.

Выше мы затрагивали термин «Медиация», обращаясь к научной статье «Медиация в коллективных трудовых спорах: ограничения и возможности (с учетом зарубежного опыта)», в Соединенных Штатах Америки медиация была впервые задействована в связи с забастовкой еще в 1898 г., когда Erdman Act зафиксировал систему регулирования споров между организациями — железнодорожными перевозчиками и работниками, обязывая стороны использовать посредничество в целях примирения. 
В 1947 г. была создана Федеральная служба США по медиации и примирению, действующая до настоящего времени. В задачи данной Службы входит предоставление посреднических услуг с целью урегулирования коллективных трудовых споров. Служба участвует в среднем в 1000 споров в год, одной из сторон спора являются различные ведомства, например Департамент сельского хозяйства США, Министерство образования США, Почтовая служба США, Департамент здравоохранения и социальных служб США и т. д. Подобные службы созданы и в отдельных штатах. Немаловажной является бесплатность услуг таких служб, финансируемых государством. Только к концу XX в. в США медиация вошла в сферу индивидуальных трудовых споров. Сегодня медиация является предпочтительным методом по сравнению с остальными процедурами АРС, о чем свидетельствуют опросы среди крупнейших корпораций. По мнению профессора Университета Ричмонда A. Ходжес, в США медиация существенно повлияла на трансформацию профсоюзов за счет повышения влияния коллективных трудовых отношений. 
Необходимо отметить высокие требования, предъявляемые к таким медиаторам в США. В частности, они должны знать принципы и практику медиации в сфере труда, принципы и методы ведения коллективных переговоров, законы штатов о труде, проблемы управления персоналом, проблемы производственных отношений, историю индустриализации и рабочего движения, экономические проблемы управления и т. д.\footnote{Филипова Ирина Анатольевна - Медиация в коллективных трудовых спорах: ограничения и возможности (с учетом зарубежного опыта) // Журнал российского права. 2018. №5 (257).}

\section{Трудовые споры в Японии}

Япония - одна из самых успешных экономик мира и стран, где трудовые отношения регулируются на уровне законодательства. В этой стране существуют множество нормативных актов, которые регулируют права и обязанности работников и работодателей. Однако, как и в любой другой, в ней возникают трудовые споры. Рассмотрим особенности решения трудовых споров в Японии. Основная часть- это юридические основы решения трудовых споров в Японии. 
В стране существует система защиты прав работников, которая включает в себя законы, 
а также регуляторные органы и судебную систему, 
которые регулируют трудовые отношения между работниками и работодателями. 
Решение трудовых споров регулируется Трудовым законодательством Японии, 
которое регулирует трудовые отношения между работниками и работодателями. 
Система трудового законодательства в Японии устанавливает правила по оплате и 
увольнению работников, а также механизмы защиты прав работников. 
Cуществует механизм постановки на рассмотрение трудовых споров, 
который регулируется Законом о разрешении трудовых споров 
(Labour Dispute Resolution Act).
 Этот акт был принят в 1947 году и подвергался не однократно изменениям. 
 Согласно Закону о разрешении трудовых споров, работник или группа работников может 
 обратиться к суду с просьбой о рассмотрении трудового спора, если работник не согласен 
 с решением работодателя. Трудовые споры в Японии могут рассматриваться как в суде по 
 делам о трудовых спорах, так и в комиссии по их разрешению, так как в Японии 
 существуют органы по разрешению трудовых споров - инспекции труда. 
 Это государственные органы, которые занимаются проверками условий труда и м
огут вынести решения по спорам, связанным с нарушением законодательства в области труда.
  По законодательству Японии, работник имеет право на правовую защиту в случае, 
  если его права были нарушены в рамках трудового договора. В суде по делам о трудовых 
  спорах в Японии рассматриваются такие вопросы, как возмещение ущерба, причиненного 
  работодателем в результате нарушения трудового договора, а также расторжение трудового 
  договора. Помимо судебной системы в Японии существуют альтернативные методы решения 
  споров, такие как медиация, переговоры и посредничество. Эти механизмы позволяют 
  работодателям и работникам достигать соглашений без обращения в суд, что существенно 
  экономит время и затраты. Отдельно стоит отметить, что в Японии широко распространены 
  профсоюзы, которые являются сильными и активными в защите прав работников. 
  Система профсоюзов в Японии охватывает почти все отрасли экономики, и профсоюзы вносят 
  значительный вклад в общее улучшение условий труда в стране. Но здесь стоит, отметить важную особенность. Всё дело в подходе диалога между профсоюзами и работодателями, рассмотрим на примере сравнения поведения профсоюзов США и Японии в период 70-х и 50-х годов, в сфере автомобильной промышленности. В 70-х годах проф союзы буквально уничтожали производства путём байкоторивания на предприятиях. По итогу всё вылилось в то что у глав компаний ни осталось выбора кроме как пойти на поводу у защитников прав рабочих, возникает вопрос а что в этом такого? Ведь идёт отстаивание прав обычных работяг, но стоит отметить что если условный рабочий постоянно пьёт, делает брак на производстве, систематически не является на рабочее место то это очень ценный сотрудник и его ни при каких обстоятельствах нельзя было увольнять, конечно уволить его было возможно но только после одобрения профсоюза, и как можно догадаться последний постоянно отвечал отказом, такая политика привела к полному упадку Детройта в 80-х годах, где как раз таки и была сконцентрирована вся автомобильная промышленность Америки того времени. А теперь посмотрим на Японию, 50-е годы для Японии были очень трудными, страна в руинах, люди работали практически за еду и когда автомобильная промышленность только вставала на ноги, профсоюзы начали «открывать рот», но в отличие от американских коллег, японские повально увольняли нерадивых сотрудников тысячами, они понимали что если сейчас они дадут волю защитникам прав работников то ни о каком успехе не может идти речи. И буквально за 20 лет к 70м годам японская автомобильная промышленность выходит на рынок Америки, и если накинуть на эту ситуацию топливный кризис 70-х годов то японские маленькие автомобили с их малолитражными и надежными двигателями выглядели как сокрушающий удар автомобильным брендам США, в это время строится рядом  с Детройтом автомобильный завод HONDA, которая представляет огромную угрозу автомобильному рынку США, и как только GM понял что проигрывает, он не придумал ничего лучше как выкупать некоторые японские автомобили и давать обычным людям ударить их кувалдой. И это действительно не шутка, вместо того что бы как то налаживать производства, изобретать что то новое, система американских профсоюзов пошла по другому пути. Отличие японских и американских систем защиты прав заключено в том, что в японской версии, существует диалог между сторонами и одна общая цель, сделать компанию как можно лучше.

Таким образом, решение трудовых споров в Японии основывается на законодательстве и судебной системе, а также на работе профсоюзов, которые защищают права работников и способствуют улучшению условий труда. В Японии имеются альтернативные методы решения споров, такие как медиация, переговоры и посредничество, которые позволяют работодателям и работникам достигнуть соглашений без обращения в суд, что существенно экономит время и затраты. Механизмы решения трудовых споров в Японии являются важным инструментом для защиты прав работников и улучшения условий труда в стране.
\section{Трудовые споры во Франции}
В настоящее время во Франции существует ряд проблем, связанных с решением трудовых споров. Некоторые специалисты указывают на то, что число трудовых споров увеличивается каждый год, что свидетельствует о том, что в данной области необходимы изменения и усовершенствования. Мы рассмотрим современную систему решения трудовых споров во Франции, а также выявим основные проблемы и предложим возможные пути их решения. Особенности современной системы решения трудовых споров во Франции. Во Франции существует целый ряд органов и институтов, задача которых состоит в решении трудовых споров. Среди них можно выделить следующие: 

- Комиссии по урегулированию споров на местах; 

- Комиссии по урегулированию споров на общенациональном уровне;

- Арбитражные суды; 

- Суды по трудовым спорам; 

- Административные суды. 

Как правило, первым этапом урегулирования трудовых споров является обращение к комиссии по урегулированию споров на местах. Эта комиссия представляет собой неформальную группу людей, которые занимаются рассмотрением проблем, связанных с трудовыми отношениями. Если же решить спор на местах не удается, то стороны могут обратиться к комиссии по урегулированию споров на общенациональном уровне. Эта комиссия имеет более серьезный статус и широкие полномочия по рассмотрению такого рода проблем. Если и решить трудовой спор на общенациональном уровне не удается, то эту задачу берут на себя арбитражные суды. Как правило, арбитражные судьи назначаются на период нескольких месяцев и имеют достаточно серьезный опыт работы в данной области. Однако, несмотря на то, что во Франции в системе решения трудовых споров присутствует большое количество институтов и органов, урегулировать всех споры не всегда удается.

Основные проблемы системы решения трудовых споров во Франции. Одной из наиболее значимых проблем, связанных с решением трудовых споров во Франции, является слишком большое количество инстанций, которые занимаются такого рода проблемами. Именно поэтому на решение одного и того же спора может уходить большое количество времени и не всегда удается добиться желаемого результата. Еще одной проблемой является отсутствие единой системы классификации трудовых споров. Как правило, каждая инстанция использует свое видение проблемы и не всегда это видение совпадает с мнениями других инстанций. 

Возможности усовершенствования системы решения трудовых споров во Франции. Одним из основных путей решения проблем, связанных с решением трудовых споров во Франции, является объединение различных инстанций, занимающихся решением такого рода проблем. Такое объединение позволит упростить систему решения споров и сократить число инстанций. Кроме того, необходимо разработать единую классификацию трудовых споров, которая была бы принята всеми инстанциями, занимающимися решением такого рода проблем. 

Таким образом, система решения трудовых споров во Франции имеет свои преимущества и недостатки. Однако, несмотря на некоторые проблемы, существующие в данной области, можно выделить возможности для усовершенствования данной системы. Важно понимать, что решение этих проблем требует совместных усилий правительства, работодателей и профсоюзов.

Таким образом, можно сделать вывод, что реформирование систем защиты трудовых прав и процедур разрешения трудовых споров - важный элемент развития общества. При этом, каждая страна должна разрабатывать свои собственные методы на основе мирового опыта и местных особенностей. Существуют различные способы решения трудовых споров, начиная от рассмотрения в суде до медиаторских и арбитражных процедур, но все они направлены на защиту прав работников и обеспечение социальной стабильности.

\chapter{ИСТОРИЧЕСКИЙ АНАЛИЗ ТРУДОВЫХ СПОРОВ В РОССИИ}
\section{История решения трудовых споров в России.}
Трудовые споры возникают в процессе трудовых отношений между работниками и работодателями. Они связаны с нарушением условий труда, низкой оплатой труда, неправомерным увольнением, дискриминацией и другими причинами. История решения трудовых споров связана с развитием обще-ственно-политических и экономических процессов.

С начала 1990-х годов в России произошли кардинальные изменения в трудовых отношениях. Рыночная экономика и приватизация привели к изменению существующей системы трудовых отношений. Работники стали более подвижными, а работодатели – более ответственными за соблюдение законодательства. Однако это также привело к увеличению числа трудовых споров, связанных с неудовлетворительными условиями труда, низкой оплатой труда и другими причинами.

В современной России трудовые споры решаются в судебных органах и арбитражных судах. Профсоюзы также участвуют в решении трудовых споров через коллективные договоры, переговоры и защиту прав работников. Важным элементом решения трудовых споров стало участие государственных инспекторов труда, которые контролируют соблюдение прав работников и выявляют нарушения законодательства в области труда.Таким образом, история решения трудовых споров связана с развитием общественных, экономических и политических процессов в России. За последние годы были созданы механизмы, обеспечивающие защиту прав работников и регулирование трудовых споров. Однако проблемы все еще существуют, и контроль за соблюдением законодательства в области труда является актуальной задачей для государства и общественных организаций.

\section{Решение трудовых споров в Древней Руси}
Споры между работниками и работодателями в Древней Руси возникали по разным причинам: начиная от неоплаты труда и заканчивая нарушением трудового законодательства. Особенно много таких споров возникало в отношениях мастеров и учеников. Ученики, как правило, были несовершеннолетними, не имели своего достатка, питались за счет мастера и получали умение, на котором в будущем должны были зарабатывать жизнь. В свою очередь мастер имел право требовать от ученика точного исполнения своих обязанностей и был ответственен за его воспитание, обучение и заработную плату. Одним из примеров трудовых споров в Древней Руси было нарушение крестьянами прав на землю. В России существовала система крепостного права, когда крестьяне были зависимы от своих помещиков и не имели права сменившись местом жительства самому выбирать работу и зарабатывать на жизнь. В связи с этим владельцы земель часто нарушали права крестьян и вынуждали их работать на своих усадьбах без оплаты. Крестьяне могли обратиться в суд и требовать справедливости, но в большинстве случаев такие конфликты решались насильственным путем и на самом деле оно понятно кто такой крепостной крестьянин и его барин? У последнего огромые связи и знакомства и поэтому почти в 100\% случаях суд вставал на сторону хозяина крестьянина. Вообще очень трудно говорить о каких то трудовых спорах в ситуациях когда крестьянами торговали, проигрывали в карты, отдавали в качестве долгов, залогов и т.д.. В заключение можно отметить, что трудовые споры были распространены в Древней Руси не сильно, о чем можно говорить если юольшинство крепостных читать и писать не умели, не говоря о знании закорнов. Некоторые из конфликтов решались с помощью суда, но большинство были разрешены насильственным путем. Однако, с развитием общества и изменением экономических условий работы, в России возникли новые виды трудовых споров, которые в настоящее время разрешаются в соответствии с законодательством.

\section{Решение трудовых споров в Российской Империи}

Одни из первых попыток решений трудовых споров в Российской Империи появились в конце XIX века. В 1897 году была создана Всероссийская ассоциация фабричной инспекции (ВАФИ), которая занималась контролем за условиями труда на фабриках, а также разрешением трудовых споров. Однако ВАФИ была скорее неким противоречием реальных отношений между работничеством и фабричными владельцами, чем полноценным регулятором. В 1906 году, после Столыпинских реформ, были созданы Органы по разрешению трудовых споров, а в 1912 году Комиссия по разрешению трудовых споров. Однако и эти институты не могли обеспечить реальное разрешение споров, так как их решения были необязательными для сторон конфликта. Окончательно ситуация изменилась в 1917 году, когда в России началась революция. Во время Октябрьской революции была учреждена Всероссийская центральная исполнительная комиссия (ВЦИК), которая начала выполнять функции по разрешению трудовых споров. В Январе 1918 года был принят декрет "О труде", в котором были установлены права работников и работодателей, а также процедура разрешения трудовых споров. Согласно этому законодательству для разрешения споров были созданы трехсторонние комиссии, состоящие из представителей работников, работодателей и государства. Решения комиссий были обязательными для выполнения обеими сторонами. Это законодательство стало основой для регулирования трудовых споров в России и было большим шагом в развитии социально-экономических отношений в стране. В заключение Трудовые споры всегда были и будут актуальным вопросом в любой стране. В Российской империи решение этих споров было достаточно сложным и долгим процессом, стоит так же признать что Российская империя отставала, по многим фронтам, например в той же промышленности, в то время как Европпа и США во всю занимались освоением метала в Российской Империи ещё занимались обработкой дерева, в 1883 году во Франции появилась первая автомобильная компания, в германии в 1886, а в России самостоятельный не лицензированный, разработанный самостоятельно автомобиль изобретут лишь в 70-х годах. 

\section{Решение трудовых споров в СССР}

Трудовые споры на предприятиях и в организациях являлись одной из наиболее острых проблем в период существования СССР. Несмотря на попытки государства создать равные условия для трудящихся и работодателей, конфликты между ними неизбежно возникали. В данной части работы, будет рассмотрено решение трудовых споров в СССР. Нормативно-правовая база решения трудовых споров. Ключевыми законодательными актами СССР, регулирующими процедуру разрешения трудовых споров, были: 

- Конституция СССР;

- Трудовой кодекс РСФСР;

- Закон о профессиональных союзах РСФСР; 

- Положение о порядке разрешения трудовых споров. 

Согласно Конституции СССР, каждый работник имел право на защиту своих трудовых прав и интересов. В свою очередь, работодатель обязан был обеспечивать необходимые условия труда и выполнять все свои обязательства перед работниками в соответствии с законодательством. Трудовой кодекс РС-ФСР определял порядок разрешения индивидуальных и коллективных трудовых споров. В соответствии с этим законодательным актом, каждый работник имел право на обращение в профсоюз, администрацию предприятия, а также в суд в случае нарушения его трудовых прав. Закон о профессиональных союзах РСФСР определял права и обязанности профсоюзов, а также порядок регистрации и организации их деятельности. Одним из главных направлений профсоюзной деятельности было разрешение трудовых споров. Положение о порядке разрешения трудовых споров устанавливало процедуру рассмотрения индивидуальных и коллективных трудовых споров в различных органах и уровнях государственной власти и управления. 

Органы разрешения трудовых споров На различных уровнях государственной власти и управления были созданы органы, которые выполняли функции по разрешению трудовых споров. На первом уровне были органы внутрипредприятийского управления, такие как профкомы и комиссии по трудовым спорам. Они рассматривали индивидуальные трудовые споры, возникающие между работодателем и отдельным работником. На втором уровне были областные и республиканские комиссии, которые занимались коллективными трудовыми спорами и рассматривали большинство обращений, которые не могли быть разрешены на уровне предприятия. На третьем уровне находился Главный арбитражный суд СССР, который был ответственен за рассмотрение самых сложных и значимых случаев трудовых споров. На этом уровне оспаривались решения вышестоящих комиссий и судебных инстанций.

Процедура разрешения трудовых споров начиналась с обращения работника или профсоюза в орган, компетентный для решения данного спора. Если индивидуальный трудовой спор не мог быть разрешен на уровне предприятия, дело передавалось в комиссию по трудовым спорам вышестоящего уровня. В случае коллективного трудового спора обращение рассматривалось на соответствующем уровне органов управления. После рассмотрения дела орган выносил решение, которое могло быть обжаловано в вышестоящий орган в течение определенного срока. В случае невозможности разрешения трудового спора в добровольном порядке, дело передавалось в суд. 

В России до революции 1917 года трудовые споры решались с помощью представителей церкви, государства и общественных организаций. Но работники не имели возможности защитить свои права, так как их профсоюзы были запрещены. После Октябрьской революции были созданы профсоюзы, которые стали защищать интересы работников. Они стали активно участвовать в решении трудовых споров и сильно повлияли на улучшение условий труда. Однако в момент свержения императора временным правительством был подписан знаменитый приказ №1. Приказом предписывалось немедленно создать выборные комитеты из представителей нижних чинов во всех воинских частях, подразделениях и службах, а также на кораблях. Главным в Приказе № 1 был третий пункт, согласно которому во всех политических выступлениях воинские части подчинялись теперь не офицерам, а своим выборным комитетам и Совету. В приказе предусматривалось также, что всё оружие передается в распоряжение и под контроль солдатских комитетов. Приказом вводилось равенство прав «нижних чинов» с остальными гражданами в политической, общегражданской и частной жизни, отменялось титулование офицеров.»; грубо говоря, выборные комитеты -- это те самые профсоюзы, а передача полной власти в их руки -- это ни что иное, как ситуация с профсоюзами GM, и что там это ничем хорошим не вышло, что во время первой мировой российская армия в один миг подорвалась изнутри, солдаты сами решали идти им в бой или нет, рыть окопы или идти на ужин. Но благо руководство страны это поняло быстро и в кратчайшие сроки исходя из ситуации в стране приняло соответствующие меры.

В послевоенное время в СССР было создано много законодательных актов, регулирующих трудовые отношения и защищающих права работников. С 1960-х годов началось развитие коллективных договоров, которые являются основой регулирования трудовых споров. Они устанавливают правила оплаты труда, графики работы, социальные гарантии и другие условия труда. Важным этапом в решении трудовых споров стало создание судебных органов, рассматривающих трудовые дела.


В период существования СССР была создана широкая нормативно-правовая база по разрешению трудовых споров, она существенно отличалась от системы царской империи и тем более древней Руси, кроме этого советская система была в некоторых моментах даже лучше, чем система запада.  Существовало достаточное количество органов государственной власти и управления на всех уровнях, ответственных за решение трудовых конфликтов. Однако в практике борьбы с трудовыми спорами были выявлены определенные недостатки, такие как длительность процедуры разрешения, недостаточная эффективность некоторых органов, низкий уровень компетенции некоторых специалистов. В целом, решение трудовых споров в СССР и современной Российской Федерации имеет многие общие черты, но и существенные отличия. В современную эпоху упор делается на урегулирование споров в рамках договора между работником и работодателем, без обращения к органам государственной власти и управления.

\newpage
\conclusion

Трудовые споры – это неизбежная составляющая трудовых отношений. Они могут возникать по самым разным причинам и связаны с нарушением трудовых прав и интересов работников и работодателей. Эволюция прав работника и работодателя не останавливается, она развивается каждый день. Мы видим определённые преимущества и недостатки в Российской системе решения трудовых спорах, мы сравниваем наш опыт и зарубежный, мы несколько отстаём от запада в этой сфере и это очень хорошо, так как мы можем увидеть ошибки, которые допустили они и подстроить некоторые их элементы под наше законодательство. 

В данной курсовой работе была разобрана система решения трудовых споров в России, причем не только в современной, а во всей её временной линии. мы рассмотрели её с теоретической, юридической и с практической точки. Мы рассмотрели системы зарубежных стран, увидели их преимущества и недостатки. На наш взгляд система решения трудовых споров в СССР и РФ схожи но они отличаются, сфера экономики и сфера рабочих отношений прямо взаимосвязаны между собой, только в советской системе экономики была заложена плановая хозяйственность, государство само решало что нужно производить сколько производить и сколько это будет стоить, а в рыночной системе на которую перешла Россия после распада СССР, правила диктует не государство а рынок, и в этой системе работодатель уже заинтересован как произвести некий продукт максимально дёшево, а работник наоборот заинтересован в том что бы ему больше платили. Существуют ситуации, когда Нарушения трудового законодательства часто встречаются и происходят еще до заключения трудового договора, когда работодатель выставляет искателю вакансии на незаконных условиях, требуя отказаться от отпуска и выходных дней, а также работать сверхурочно. Как результат, трудовой договор заключается на условиях, которые нарушают законодательство. В современной России можно отметить наличие двух правовых режимов в регулировании трудовых отношений: государственные организации соблюдают Трудовой кодекс РФ, тогда как в коммерческом секторе, особенно на малых и средних предприятиях, трудовое законодательство игнорируется в пользу гражданско-правовых отношений между работниками и работодателями. Наличие неофициальных договоренностей оставляет работников без социальной защиты и возможности защищать свои права, так как на таких предприятиях не существует профсоюзов и комиссий по трудовым спорам. Все это приводит к росту числа незащищенных и социально уязвимых работников. Такая ситуация часто приводит к возникновению индивидуальных и коллективных трудовых споров. Законодательство РФ подробно определяет процедуры разрешения таких споров, предоставляя два органа для этого - комиссию по трудовым спорам и суд. Однако, опыт показывает, что комиссия не всегда может эффективно защитить права работника, так как ее состав, состоящий как из назначенных работодателем, так и из выбранных работниками членов, может быть зависимым от работодателей. Поэтому, наиболее эффективным способом защиты своих прав является обращение в суд. В настоящей экономической обстановке работники не рискуют открыто выступать за свои права и не вступают в конфликт с работодателем, так как подача жалобы может привести к потере работы. Поэтому, зачастую, пострадавшие имеют больше выгоды от воздержания от применения своих прав, чем от вступления в спор. Когда работник обращается в суд, например, для восстановления на работе или получения оплаты за неоплачиваемые дни, работодатель рассматривает это как нежелательное и ненормальное поведение, и может преследовать судебным порядком. Такая ситуация не соответствует принципам Конституции РФ о равенстве перед законом и защите прав человека. Один из заметных фактов заключается в том, что незаконно уволенные работники не всегда обращаются в суд с требованием о восстановлении на работе. Они понимают, что могут быть уволены вновь, теперь по другому основанию. Поэтому, было бы целесообразно внести в Трудовой кодекс РФ положения, предусматривающие выплату денежной компенсации вместо восстановления на работе. Для того чтобы защитить права работников, необходимо улучшить судебную практику и решать правовые вопросы, а также отслеживать соблюдение трудовых прав и доводить результаты контроля до законодателя, который должен принимать меры, позволяющие обеспечить выполнение норм с отрицательным правоприменительным балансом. Человек, желающий защитить свои права, обращается в суд. Улучшение работы судов требует расширения их возможностей, изменений в законодательстве, совершенствования системы и процедур. С 1995 года суд не обязан собирать доказательства, поэтому стороны должны предоставлять свои доказательства. Это приводит к тому, что в основном защищаются те, у кого есть деньги на дорогого адвоката. Однако, правосудие в России не всегда достигает своей конечной цели, что требует создания специализированных судов для трудовых споров. Такие суды помогут сократить сроки рассмотрения дел и исполнения судебных решений, а также избавят от множественных юрисдикционных органов. Создание таких судов актуально в современных условиях, когда локальные акты регулируют трудовые отношения. Для эффективного улучшения трудовых отношений необходимо соблюдение правил и порядка на производстве, а также уважение к законам трудового законодательства. Этот процесс направлен на педагогическое обучение правилам и беспримирность к нарушениям законности, предотвращение нарушений трудовых прав и устранение их причин. Эффективная реализация этих мер помогает уменьшить и устранить причины конфликтов на рабочем месте.

